\section{Convergence in Law}%
\label{sec:convergence_in_law}

\begin{definition}
    $Y_n$ is \textit{bounded in probability} if for any $\epsilon>0\ \exists K$
    and $n_0(\epsilon)$ such that
    \begin{equation*}
        P(|Y_n| > K) < \epsilon \text{ for all } n > n_0(\epsilon).
    \end{equation*}
\end{definition}

\begin{definition}{Convergence in Law/Distribution}
    $Y_n \convl Y$ if $H_n(x) \to H(x)$ for all continuity points $x$.
\end{definition}

\begin{thm}
    If $Y_n \convl H$ then $Y_n$ is bounded in probability.
\end{thm}

\begin{thm}{Slutsky's Theorem}
    If $Y_n \convl Y$, $A_n \convp a$, and $B_n \convp b$, then
    \begin{equation*}
        A_n + B_n Y_n \convl a + b Y.
    \end{equation*}
\end{thm}

\begin{corollary}
    If $Y_n \convl Y$, $R_n \convp 0$, and $B_n \convp 1$, then
    \begin{align*}
        Y_n + R_n &\convl Y \\
        Y_n / B_n &\convl Y.
    \end{align*}
\end{corollary}

\begin{thm}
    If $k_n (Y_n - c) \convl H$ and $k_n \to \infty$ then $Y_n \convp c$.
\end{thm}

\begin{definition}
    $Y_n \convp Y$ if $Y_n - Y \convp 0$.
\end{definition}

\begin{thm}
    $Y_n \convp Y \implies Y_n \convl Y$ 
\end{thm}
